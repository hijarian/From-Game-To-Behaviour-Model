\chapter{Введение}
В данной работе будет построена и реализована математическая модель, имитирующая поведение некоторых, во всяком случае, рационально мыслящих, существ (например, человека). Также произведена оценка работоспособности предположений, при которых она может быть реализована и методов, выбранных для реализации.

Изучение поведения человека, формализация его в виде какой-либо математической модели и совершение предсказаний на её основе являются фундаментальными вопросами из нескольких смежных областей знаний, таких как психология, социология, биохимия, генетика и наука об искусственном интеллекте. Актуальность этих вопросов не исчезнет до тех пор, пока не будет найдено их гарантированное точное решение.

В основе работы лежит компьютерная игра, в том числе и как своего рода подтверждение той степени схожести компьютерных игр и формальных моделей, позволяющих производить научные эксперименты на их основе.

Здесь существует определённая терминологическая проблема, о которой следует немедленно сделать замечание.

Дело в том, что интерактивные симуляции, разработанные для выполнения на той или иной вычислительной технике, принципиально неверно называть ``компьютерными играми'', несмотря на то, что такова распространённая среди обывателей и СМИ практика. Данное терминилогическое противоречие фактически очевидно, хотя подробное исследование на эту тему было бы полезно. Везде в дальнейшем вместо привлекательно краткого термина ``компьютерная игра'' будет употребляться более точный ``интерактивная симуляция''. Пользователя симуляцией за неимением лучшего термина будем называть ``игроком''.

Во второй главе работы будет рассказано о первоисточнике исследования.

В третьей главе будет формально поставлена задача, сформулированы необходимые ограничения и допущения, а также утверждена терминология, которая будет использована в остальной работе.

В четвёртой главе будет описан процесс построения реализации модели и результаты проведённого тестирования.

Заключение традиционно содержит выводы и предположения о сферах применения данной работы и способах дальнейшего развития темы.
