\chapter{Первоисточник работы}
В 1998 году японская фирма GAINAX выпускает интерактивную симуляцию под названием ``Princess Maker''. Данная симуляция изображает воспитание человеческого ребёнка --- 10-летней девочки --- взрослым самодостаточным мужчиной в условиях города, в вымышленном фантастическом мире. 

Участие игрока в симуляции заключается в определении того, чем девочка будет заниматься, последовательно, на протяжении условных 8 лет каждый месяц, составляя расписание на каждые 10 дней этого месяца. Таким образом, игрок обязан за время взаимодействия с симуляцией совершить $8\times12\times 3=288$ выборов дальнейшего действия подопечной девочки, $8\times 12=96$ раз по три выбора за раз.

Условная подопечная концептуально представлена в виде набора целочисленных именованных параметров числом 29, для удобства игрока разделённых на группы по семантическому сходству.

После указания игроком очередных трёх действий, которые следует совершить, симуляция производит расчёт изменения параметров и предоставляет игроку отчёт о прошедших изменениях вместе с новыми значениями параметров.

Предполагается и ожидается, что игрок совершает выбор дальнейших действий подопечной на основании текущих значений её параметров и знаний о том, как то или иное действие влияет на значения параметров. Подразумевается, что для игрока определённые значения параметров подопечной (или по крайней мере их сотношение) являются предпочтительными, и, вследствие этого, в своём выборе он также руководствуется общей стратегией, благодаря которой, на взгляд игрока, параметры вернее всего примут желаемые значения.

По окончании последовательности выборов действий и связанных с ними изменений значений параметров подопечной игрока симуляция формирует итоговый отчёт, развёрнуто в нарративной форме описывая конечное состояние особи. Симуляция условно производит оценку с позиций достигнутого социального положения, морально-этического характера действий и отношений между воспитуемой и воспитателем. Всё множество конечных состояний особи благодаря специфике формулировки отчёта разбито на конечное число вариантов окончания симуляции. Это число порядка $10^3$.

Данная интерактивная симуляция является примером подробной и вместе с тем лаконичной интерактивной модели развития субъекта поведения, одновременно являющегося объектом воздействия последствий совершаемых действий. Конечно же, любая интерактивная симуляция, которая может быть отнесена к жанру ``компьютерных ролевых игр'', является в точности такой моделью. Однако процесс ``Princess Maker'' лишён большей части характерных для жанра примесей и состоит только из ключевых событий: дискретизированного выбора действий, описания следующих за выбором изменений и описания конечного состояния субъекта поведения. Данная симуляция беспрецедентна и по состоянию на июнь 2010 года уникальна для индустрии электронных развлечений.

