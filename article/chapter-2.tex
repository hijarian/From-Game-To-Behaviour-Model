\chapter{Постановка задачи. Формализация процессов. Упрощения}
Имея функционирующую интерактивную модель развития субъекта поведения, возможно сформулировать следующую задачу.

Найти последовательность действий, которая приведёт субъект поведения к желаемому конечному состоянию.

Следует учитывать значительные размеры дерева решений данной задачи для рассматриваемой модели: на каждом из 288 шагов выбора требуется выбирать из $\sim 30$ альтернатив. Значит, на 100-м шаге дерево выбора содержит $30^{100}$ состояний.

Данное обстоятельство позволяет немедленно отвергнуть традиционные методы поиска по дереву решений в пользу каких-либо иных.

Примем некоторые соглашения о терминологии.

